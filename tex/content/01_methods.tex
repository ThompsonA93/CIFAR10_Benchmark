\chapter{Methods and Resources}
The datasets are imported from a remote destination with the use of the \href{https://keras.io/api/datasets/cifar10/}{keras.datasets} package.
Beside additional utility libraries, such as \href{https://matplotlib.org/}{matplotlib} and \href{https://seaborn.pydata.org/}{seaborn} for displaying graphics, the \href{https://www.tensorflow.org/}{Tensorflow} library supplies the necessary algorithms for creating and executing neural network computations and hyperparameter evaluation.
A secondary approach using \href{https://scikit-learn.org/stable/modules/generated/sklearn.neural_network.MLPClassifier.html}{SKLearn multi-layer perceptron} serves as additional reference in terms of computational performance and use. Notably, SKlearn implements \href{https://scikit-learn.org/stable/modules/generated/sklearn.model_selection.GridSearchCV.html}{GridSearch} for parameter search.

\subsubsection{Approach}
After the download of the CIFAR-10 data, it is required to perform some data transmutations on the set.
The data values are normalized from the value range of (0..255) to (0..1), by simple division. 
For the Keras computations, the training and test labels are categorized from a class vector of integers to a binary class matrix.
SKLearn requires a distinct approach by reshaping the available data of the form (Images, X-Pixel, Y-Pixel, RGB-Channel) to (Images, \{X-Pixel, Y-Pixel, RGB-Channel\}).
Optionally, the size of the used data is reduced in order to reduce the time necessary to run analytics.
Afterwards, the main computation on the data commences.
